\documentclass{report}

% Packages
\usepackage[dutch]{babel}
\usepackage{datetime}
\usepackage{etoolbox}
\makeatletter
\patchcmd{\@makechapterhead}{50\p@}{0pt}{}{}
\patchcmd{\@makeschapterhead}{50\p@}{0pt}{}{}
\makeatother
\usepackage[nottoc,numbib]{tocbibind}
\usepackage{hyperref}
%\hypersetup{
%  colorlinks=true,
%  linkcolor=black,
%  urlcolor=blue
%}
\usepackage{graphicx}

% Literals
\newcommand{\versionnumber}{1.0}

\newcommand{\name}{Swen Meeuwes}
\newcommand{\studentnumber}{0887127}
\newcommand{\email}{0887127@hr.nl}
\newcommand{\mobilephone}{06 10 466 433}

\newcommand{\institution}{Rotterdam University of Applied Sciences}
\newcommand{\organisation}{\&ranj}

\renewcommand{\title}{Narratieve omgevingsbewerkers in serious games}
\newcommand{\subtitle}{}

\begin{document}
\begin{titlepage}
        \centering
        \includegraphics[width=2cm]{Images/University}\par
        \vspace{4\baselineskip}
        {\Huge\title\par}
        {\Large\subtitle\par}
        \par
        \includegraphics[width=3cm]{Images/Organisation}
        \vspace{4\baselineskip}
        \par
        {\Large\name\par}
        {Studentnummer: \studentnumber\par}
        \vfill
        {\hfill Versie: \versionnumber\par}
        {\hfill \today}
\end{titlepage}

\chapter*{Samenvatting}
Dit document beschrijft het voorstel voor het afstudeeronderzoek met de titel ``\emph{Het opzetten van narratieve omgevingsbewerkers in narratieve serious games}''.

Het \emph{serious game} en \emph{gamification} bedrijf \organisation{} past games toe om positieve gedragsverandering te bevorderen \cite{websiteranj}. Hiervoor maken ze gebruik van narratieven in games; de context van het spel is verwerkt in een verhaal dat bestaat uit verschillende dialogen. De speler doorloopt het verhaal en maakt keuzes in de dialogen. Aan het einde van het interactieve narratief krijgt de speler direct een evaluatie over de keuzes die gemaakt zijn tijdens het spelen. Een voorbeeld van zo'n spel is Fair Play \cite{fairplay}. Dit spel maakt de speler bewust van discriminatie in het hedendaagse leven en leert deze er mee om te gaan.

Achter deze games zit een engine die narratieve data uit de bewerkers van \organisation{} kan interpreteren. Echter zijn de bewerkers verouderd en voldoen niet meer aan de eisen en wensen van haar gebruikers; de game designers en game developers \cite{interviewivo}. Dit zorgt voor een belemmering op het gebied van effici{\"e}ntie en innovatie. Zo zijn de type nodes (ook wel content types genoemd) in dit node-based system ingebakken in de engines en de bewerkers \cite{interviewivo}. Dit maakt het lastig om nieuwe content types toe te voegen. Verder zorgen de verschillende content types voor vervuiling binnen de bewerkers en de engine omdat er slechts enkele types worden gebruikt in ieder spel. Tenslotte zijn de bewerkers gemaakt in een verouderde SDK en programmeertaal \cite{interviewivo}. Er zijn weinig programmeurs binnen \organisation{} die nog kennis hebben van deze SDK en programmeertaal \cite{interviewivo}. Dit alles maakt de bewerkers slecht schaalbaar en moeilijk te onderhouden.

Het doel van dit onderzoek is om zoveel mogelijk kennis en ervaring te verzamelen op het gebied van narratieve omgevingsbewerkers. Deze informatie kan \organisation{} gebruiken als basis voor de toekomstige bewerkers. Hiervoor is de volgende centrale onderzoeksvraag opgesteld: ``Hoe kan er een schaalbare narratieve omgevingsbewerker worden opgezet die inzetbaar is voor verschillende projecten en te hanteren is door haar gebruikers?''. Narratieve omgevingsbewerkers zijn hier de bewerkers voor het verhaal en haar dialogen. De gebruikers van deze bewerkers zijn de game designers en game developers van \organisation.

Om deze onderzoeksvraag te beantwoorden zullen er interviews afgenomen worden met de gebruikers van de narratieve omgevingsbewerkers en de technical team lead van \organisation. Verder zal er literatuuronderzoek gedaan worden naar schaalbare applicaties, dataformaten en visual scripting. Eventueel vervolgonderzoek zou zich kunnen richten op het interpreteren van de ge{\"e}xporteerde data afkomstig van de bewerkers.

Het onderzoek zal worden afgenomen in samenwerking met \organisation. Dit serious gaming bedrijf bestaat uit ongeveer 50 medewerkers waarvan 7 vaste IT-medewerkers \cite{linkedinranj}. In het onderzoek komen de vakgebieden consultancy, software engineering, datastructuren and mogelijk algoritmes binnen de IT naar boven.

De oplevering aan het einde van dit onderzoek zal bestaan uit een adviesrapport met eventueel bijgeleverde prototypes. Verder zal elk prototype met een technisch ontwerp worden aangeleverd.

\newpage

\tableofcontents

\chapter{Inleiding}
\section{Werktitel}
Het opzetten van narratieve omgevingsbewerkers in narratieve \emph{serious games}.

\section{Aanleiding} % narrative omgevingsbewerker is een node-based system met contenttypes....definitie bewerker
%Moet naar over bedrijf
%Al sinds 1999 specialiseert \organisation zich in \emph{serious games}. Hoewel er meerdere definities van \emph{serious games} zijn beschrijven ze allemaal een spelomgeving waarin de focus niet uitsluitend op entertainment ligt \cite{Tarja2007}. Hiernaast komen bij het begrip \emph{serious games} woorden zoals 'training', 'simulatie', 'reclame' en 'educatie' naar boven \cite{Tarja2007}.
%\organisation past \emph{serious games} vooral toe in educatie, trainingen en simulaties door middel van een narratief \cite{websiteranj}.
%Om workflow te bevorderen wordt er gebruik gemaakt van een dialoog bewerker \cite{interviewivo}. Echter sluit deze tool niet meer aan bij de werkwijze van \organisation \cite{interviewivo} wat zorgt voor een obstakel in het ontwikkelproces.

Rond 2008 begon \organisation{} narratieven te verwerken in hun \emph{serious games} om op verhalende wijze gedragsverandering toe te passen. Om game designers deze narratieven te laten defini{\"e}ren zijn er twee bewerkers opgezet; {\'e}{\'e}n voor de verhaallijn en {\'e}{\'e}n voor de dialogen die plaats vinden in deze verhaallijn. De huidige versies van deze bewerkers zijn gemaakt met behulp van de \href{https://en.wikipedia.org/wiki/Apache_Flex}{Apache Flex SDK} en \href{http://www.adobe.com/devnet/actionscript/articles/actionscript3_overview.html}{ActionScript3} \cite{interviewivo}.
Over de jaren heen zijn de verwachtingen van de bewerkers veranderd, maar ze zijn niet tot weinig uitgebreid omdat de achterliggende softwarearchitectuur niet schaal- en houdbaar is \cite{interviewivo}. Tenslotte werken er steeds minder programmeurs bij het bedrijf die kennis hebben van de code base achter de bewerkers.

\section{Belang}
De conclusie en de aanbevelingen die voort komen uit het afstudeerverslag zijn input voor de beslissingen binnen \organisation{} op het gebied van narratieve omgevingsbewerkers. Vanuit deze input zal het bedrijf een nieuwe omgeving opzetten waarin game designers hun narratieven kunnen defini{\"e}ren. Het is daarom ook belangrijk om de game designers bij dit onderzoek te betrekken. Verder zullen toekomstige games de output van deze narratieve bewerker moeten verwerken. Dit betekend dat de nieuwe omgeving ge{\"i}ntegreerd zal moeten worden met het game framework van \organisation{} en dat dus de game developers inspraak moeten over technische keuzes binnen het project. Tenslotte halen de klanten van \organisation{} ook profijt uit deze nieuwe narratieve bewerkingsomgeving. De nieuwe omgeving maakt de games stabieler en goedkoper om te produceren.
Verder worden de \emph{narrative games} van \organisation{} gebruikt om mensen te trainen in \emph{peacebuilding} \cite{missionzhobiaorg}. Dit draagt bij aan het verbeteren van mensenlevens in landen waarin conflict heerst. 
Ook wordt dit type game gebruikt om kinderen bewust te maken van discriminatie \cite{fairplay}.

\section{Doelstelling} %datum absoluut maken
Het bedrijf hoopt na zes maanden te beginnen met het ontwikkelen van een nieuwe narratieve omgevingsbewerker, zodat ze effici{\"e}nter en voor lagere kosten producten kunnen opleveren aan de klant. Hiervoor is het belangrijk om binnen de zes maanden zoveel mogelijk kennis en ervaring te verzamelen. Verder kan er nagedacht worden over mogelijke oplossingen op problemen die voort komen uit het onderzoek zodat deze het ontwikkelproces later niet zullen hinderen.

\section{Probleemstelling}
Voor het defini{\"e}ren van dialogen in \emph{narrative games} gebruikt \organisation{} verouderde bewerkers die gemaakt zijn met behulp van de \href{https://en.wikipedia.org/wiki/Apache_Flex}{Apache Flex SDK} en \href{http://www.adobe.com/devnet/actionscript/articles/actionscript3_overview.html}{ActionScript3} met \href{https://en.wikipedia.org/wiki/Adobe_Flash_Builder}{Adobe Flash Builder} als \emph{integrated development environment}. Echter werken er nog weinig programmeurs bij \organisation die kennis hebben van \href{https://en.wikipedia.org/wiki/Apache_Flex}{Apache Flex} en \href{http://www.adobe.com/devnet/actionscript/articles/actionscript3_overview.html}{ActionScript3}. Hierdoor wordt het steeds lastiger om deze bewerkers te onderhouden en uit te breiden. Verder wekt de architectuur en beperkte schaalbaarheid van de bewerkers frustratie op bij de game developers en game designers. Projecten verschillen in features en content, maar de huidige bewerkers maken het moeilijk om deze aspecten te splitsen per project. Hierdoor zitten er veel features in de bewerkers die maar {\'e}{\'e}n keer nodig waren en nu de bewerker bevuilen. Gebruikers van deze bewerkers hebben door de bevuiling steeds minder overzicht. Dit alles zorgt voor een daling in effici{\"e}ntie en innovatie.
De gewenste situatie is om te beschikken over een overzichtelijke narratieve omgevingsbewerker met een schaalbare en houdbare architectuur. In deze vernieuwde narratieve omgevingsbewerker kunnen er makkelijk nieuwe features en content worden toegevoegd. Verder kan de bewerker worden ingericht per project om vervuiling te voorkomen. Vervolgens kunnen game developers content integreren zonder deze in te hoeven bakken in de bewerker en game engine.

\newpage

\section{Centrale onderzoeksvraag en deelvragen}
\subsection{Centrale onderzoeksvraag}
Hoe kan er een schaalbare narratieve omgevingsbewerker worden opgezet die inzetbaar is voor verschillende projecten en te hanteren is door haar gebruikers?

\subsection{Deelvragen} % waarom dit belangrijk is moet voortstromen uit de probleemstelling
% Keywords:
% Visual scripting
% Content construct
% Integratie game framework - Data format export
% Personalisatie per project (modulair)
% Schaalbaarheid en houdbaarheid
\begin{itemize}
\item Wie gaan de narratieve omgevingsbewerker gebruiken en wat zijn hierbij hun eisen en wensen?
\item Hoe kan de bewerker inzichtelijk worden gemaakt voor haar gebruikers, zodat zij deze kunnen hanteren?
\item Wat zijn de mogelijkheden om de data achter een narratief te moduleren?
\item Welke mogelijke dataformaten zijn er om data van de narratieve omgevingsbewerker op te slaan en te exporteren?
\item Hoe kan de ge{\"e}xporteerde data vanuit de narratieve omgevingsbewerker ge{\"i}nterpreteerd worden door de game engine?
\end{itemize}

\section{Opdrachtgever}
De desbetreffende afstudeeropdracht wordt uitgevoerd bij \organisation{} gevestigd te Rotterdam. Het bedrijf houdt zich bezig met gedragsverandering door middel van \emph{gamification} en \emph{serious games} \cite{websiteranj}. Dit maakt het bedrijf actief in de creatieve sector.
Het bedrijf zelf bestaat uit ongeveer 50 medewerkers waarvan 7 vaste IT-medewerkers \cite{linkedinranj} en maakt deel uit van een grotere firma; \&samhoud. Enkele producten van \organisation{} zijn: \href{https://ranj.com/products#knowledge-knock-out}{Knowledge Knock-out}, \href{https://ranj.com/projects/corporate/development#mission-zhobia}{Mission Zhobia} (voor Peace Nexus), \href{https://ranj.com/projects/corporate/development#appie-aandeel}{Appie aandeel} (voor Albert Heijn) en \href{https://ranj.com/projects/education#pinpin}{PinPin} (voor Rabobank).
De visie van \organisation{} luidt: ``Together we build a brighter future'' \cite{websiteranj}.
Naast deze visie heeft \organisation{} 4 core values:
\begin{description}
\item[Playfulness] plezier en een goed humeur hebben. Spelenderwijs door het leven gaan.
\item[Intensity] passie om uit te blinken.
\item[Authenticity] durf jezelf te zijn, durf anders dan andere te zijn.
\item[Friendship] je kunt op elkaar rekenen, samen zijn we sterk.
\end{description}

\section{Werkomgeving en taken}
Tijdens de afstudeerperiode bij \organisation{} werkt de student nauw samen met het Corporate Learning team gevestigd te Rotterdam. Het Corporate Learning team is verantwoordelijk voor het ontwikkelen van \emph{serious games} die meestal narratieven gebruiken om gedragsverandering bij bedrijven te bevorderen. Voorbeelden van producten die dit team ontwikkeld heeft zijn \href{https://ranj.com/projects/corporate/development#mission-zhobia}{Mission Zhobia} en \href{https://ranj.nl/projects/corporate/development#internal-investigation-game}{Internal Investigation}. De student zal zich bezig houden met de problemen en frustraties rondom de huidige narratieve omgevingsbewerkers. Hieruit zal de student met suggesties komen om in de toekomstige bewerker deze punten te tackelen.
\begin{description} % check classroom competenties
\item[Analyseren] De eindcompetentie analyseren zal behaald worden door het inventariseren en ontleden van het probleem rondom de huidige bewerkers. Verder zal de student gebruikers van het systeem interviewen om zo meer te weten te komen over het ontwikkelingsproces van een \emph{narrative game}. Uit deze interviews zal informatie voort komen die de student zal verwerken tot bruikbare informatie om zo tot mogelijke oplossingen te komen.
\item[Ontwerpen] De eindcompetentie ontwerpen zal de student behalen door een voorstel te doen op structuur van de narratieve data achter de toekomstige bewerkers. Ook zal de student met suggesties komen voor de softwarearchitectuur achter de toekomstige bewerkers. Tenslotte zal er een advies worden gevormd betreft de interactie tussen de bewerkers en zijn gebruikers.
\item[Adviseren] Aan de eindcompetentie adviseren zal de student voldoen door onderbouwd en richtinggevend advies uit te brengen over het aanpakken van problemen rondom de huidige bewerkers. Hierbij zullen verschillende frameworks en architecturale principes ter sprake komen.
% todo: welke softwarecomponenten?
\item[Beheren] Door rekening te houden met de context binnen het afstudeertraject en het gebruik van verschillende softwarecomponenten zal de student voldoen aan de eindcompetentie beheren. Verder zal de student aangeleverde code houdbaar en schaalbaar opzetten zodat deze later is in te zien en mogelijk als basis kan fungeren.
\end{description}


\chapter{Methode}

\section{Onderzoeksmethode} % waarom literatuur onderzoek?
Tijdens de afstudeerperiode zal er vooral kwalitatief onderzoek gedaan worden om inzicht te krijgen in de achterliggende motivaties, behoeften en wensen van de gebruikers. Dit zal gebeuren in de vorm interviews en literatuuronderzoek. Verder zal de source code van de huidige bewerkers grondig worden bestudeerd om advies uit te brengen over problemen die zich hierin bevinden. 

\section{Informatie vergaren}
Om informatie over de wensen en eisen te vergaren zullen er continu interviews plaats vinden met de gebruikers van de narratieve omgevingsbewerkers. Met de verwerkte informatie uit deze interviews kan er een prototype worden opgezet. Dit prototype kan vervolgens per iteratie ge{\"e}valueerd en gevalideerd worden. Ook kunnen er \href{http://www.agilemodeling.com/artifacts/useCaseDiagram.htm}{use case diagrams} worden gemaakt uit de wensen en eisen van de gebruikers. Deze kunnen in de toekomst worden gebruikt bij het ontwikkelen van de nieuwe bewerkers.
Verder zullen er interviews met de technical team lead worden afgenomen. Hij is een expert op het gebied van \emph{narrative games} en kent de code base van de huidige bewerkers. Omdat hij de code base kent en met de huidige bewerkers heeft gewerkt weet hij ook waar de huidige problemen zitten \cite{interviewivo}. Een gesprek met hem levert zeer waarschijnlijk waardevolle informatie op.
Tenslotte zal literatuur onderzoek gedaan worden naar schaalbare en houdbare systemen. Dit kan waardevol zijn bij het opzetten van een prototype.

\section{Valideren van bevindingen}
De bevindingen die uit de interviews voortkomen zullen in een iteratief prototype gevalideerd worden. Technische aspecten zullen worden gevalideerd door de technical team lead van het bedrijf. Verder zullen de prototypes voorgeschoteld worden bij de gebruikers van de narratieve omgevingsbewerkers. Deze test dient ter validatie van de implementatie.

\section{Projectmethode}
Bij \organisation{} wordt er vooral met de Scrum en Waterval projectmethodes gewerkt. Voor dit afstudeeronderzoek zal er gebruik maken van Scrum. Door iteratief te werken wordt de kracht van Scrum benut

%Met deze projectmethode kan het prototype iteratief  ge{\"e}valueerd en gevalideerd worden. Zo 

\chapter{Resultaten}

\section{Beoogd resultaat van de opdracht}
Dit onderzoek zal resulteren in een rapport waarin advies uitgebracht wordt over het moduleren van data achter een narratief; het dataformaat van de ge{\"e}xporteerde data; hoe narratieve data ge{\"e}nterpreteerd kan worden door de game engine en hoe de bewerkers inzichtelijk gemaakt kunnen worden voor haar gebruikers. Hierbij zullen eventuele prototypes geleverd worden die een mogelijke implementatie van de adviezen zullen tonen. Deze prototypes zullen per stuk {\'e}{\'e}n probleem binnen het onderzoek tackelen en aantonen of de geadviseerde oplossing bruikbaar is. Verder zal elk prototype met een technisch ontwerp worden aangeleverd.

\section{Kwaliteitsverwachtingen}
\organisation{} maakt gebruik van 2 game engines; Unity3D en een eigen engine. De eigen engine bestaat uit 2 delen; de \emph{\organisation{} Software Library} gemaakt in ECMAScript 5 en het \emph{Narrative Game Template} wat gebouwd is in ECMAScript 5 en gebruik maakt van de \href{https://createjs.com/}{CreateJS library suite} \cite{interviewivo}. Tijdens het onderzoek moet hier rekening mee worden gehouden, zodat het resultaat toepasbaar is op deze game engines.

% Chapter: Literatuur
\bibliographystyle{plain}
\renewcommand{\bibname}{Referenties}
\bibliography{References}
\nocite{*}

\chapter{Betrokkenen}

\section*{Afstudeerder}
\begin{table}[h]
\begin{tabular}{ll}
Naam & \name \\
Studentnummer & \studentnumber \\
E-mailadres & \email \\
Mobiel telefoonnummer & \mobilephone
\end{tabular}
\end{table}

\section*{Bedrijfsbegeleider}
\begin{table}[h]
\begin{tabular}{ll}
Naam bedrijf/organisatie & \organisation \\
Naam bedrijfsbegeleider & Ivo Swartjes \\
E-mailadres & ivo@ranj.nl \\
Telefoonnummer \organisation & +31 (0) 10 21 23 101 \\
Functie/ rol & Technical Team Lead \\
Bezoekadres locatie organisate & Lloydstraat 21m \\ 
 & 3024 EA Rotterdam \\
 & The Netherlands \\
Website organisatie & \url{https://ranj.nl/}
\end{tabular}
\end{table}

\newpage

\section*{Opdrachtgever}
\begin{table}[h]
\begin{tabular}{ll}
Naam bedrijf/organisatie & \organisation \\
Naam opdrachtgever & Micha{\"e}l Bas \\
E-mailadres & michael@ranj.nl \\
Telefoonnummer \organisation & +31 (0) 10 21 23 101 \\
Functie/ rol & CEO \\
Bezoekadres locatie organisate & Lloydstraat 21m \\ 
 & 3024 EA Rotterdam \\
 & The Netherlands \\
Website organisatie & \url{https://ranj.nl/}
\end{tabular}
\end{table}

\end{document}