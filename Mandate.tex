\documentclass{report}

% Packages
\usepackage[dutch]{babel}
\usepackage{datetime}
\usepackage{hyperref}
\usepackage{graphicx}

% Literals
\newcommand{\versionnumber}{0.1}

\newcommand{\name}{Swen Meeuwes}
\newcommand{\studentnumber}{088127}
\newcommand{\email}{0887127@hr.nl}
\newcommand{\mobilephone}{06 10 466 433}

\newcommand{\institution}{Rotterdam University of Applied Sciences}
\newcommand{\organisation}{\&ranj }

\renewcommand{\title}{Narratieve omgevings bewerker}
\newcommand{\subtitle}{Sub title}

\begin{document}
\begin{titlepage}
        \centering
        \includegraphics[width=2cm]{Images/University}\par
        \vspace{4\baselineskip}
        {\Huge\title\par}
        %{\Large\subtitle\par}
        \par
        \includegraphics[width=3cm]{Images/Organisation}
        \vspace{4\baselineskip}
        \par
        {\Large\name\par}
        {Studentnummer: \studentnumber\par}
        \vfill
        {\hfill Versie: \versionnumber\par}
        {\hfill \today}
\end{titlepage}

\chapter*{Samenvatting}

\newpage

\tableofcontents

\newpage

\chapter{Inleiding}
\section{Werktitel}
Hoe kun je een narratieve omgevings bewerker flexibel opzetten? %draft

\section{Aanleiding}
Sinds 1999 specialiseert \organisation zich al in \emph{serious games}. Hoewel er meerdere definities van \emph{serious games} zijn beschrijven ze allemaal een spelomgeving waarin de focus niet alleen op entertainment ligt\cite{Tarja2007}. Hiernaast komen woorden zoals 'training', 'simulatie', 'reclame' en 'educatie' naar boven\cite{Tarja2007}.
\organisation past \emph{serious games} vooral toe in educatie, trainingen en simulaties door middel van een narratief\cite{websiteranj}.

Om workflow te bevorderen wordt er gebruik gemaakt van een dialoog bewerker\cite{interviewivo}. Echter sluit deze tool niet meer aan bij de werkwijze van \organisation \cite{interviewivo} wat zorgt voor een obstakel in het ontwikkelproces.

\section{Belang}
De conclusie en de aanbevelingen die voort komen uit het afstudeerverslag zijn input voor de beslissingen van \organisation op het gebied van narratieve omgevings bewerkers. Vanuit deze input zal het bedrijf een nieuwe omgeving opzetten waarin game designers hun narratieven kunnen defini{\"e}ren. Het is daarom ook belangrijk om de game designers bij dit onderzoek te betrekken. Verder zullen toekomstige games de output van deze narratieve bewerker moeten verwerken. Dit betekend dat de nieuwe omgeving ge{\"i}ntegreerd zal moeten worden met het game framework van \organisation en dat dus de game developers inspraak moeten over technische keuzes binnen het project. Tenslotte halen de klanten van \organisation ook voordeel uit deze omgeving. De nieuwe omgeving maakt de games stabieler en goedkoper om te produceren.

\section{Doelstelling}
Het bedrijf hoopt na 6 maanden te beginnen met het ontwikkelen van een nieuwe narratieve omgevings bewerker, zodat ze effici{\"e}nter en voor lagere kosten producten kunnen opleveren aan de klant. Hiervoor is het belangrijk om binnen de 6 maanden aandachtspunten, risico's en mogelijke oplossingen te inventariseren.

\section{Probleemstelling}
Voor het defini{\"e}ren van dialogen in \emph{narrative games} gebruikt \organisation verouderde bewerkers die gemaakt zijn in \href{http://www.adobe.com/devnet/actionscript/articles/actionscript3_overview.html}{ActionScript3} met \href{https://en.wikipedia.org/wiki/Adobe_Flash_Builder}{Adobe Flash Builder} als \emph{integrated development environment}. Weinig programmeurs bij \organisation hebben kennis van \href{http://www.adobe.com/devnet/actionscript/articles/actionscript3_overview.html}{ActionScript3} en daarom wordt het steeds lastiger om deze bewerkers te onderhouden. Verder wekt de schaalbaarheid van de bewerkers frustratie op bij de game designers en developers omdat projecten verschillen in content en de bewerkers dit niet of met veel moeite toelaten. Dit alles zorgt voor een daling in effici{\"e}ntie en bekommering op het gebied van innovatie.
In de ideale situatie wil \organisation per project de bewerkers kunnen inrichten en uitbreiden zodat er effici{\"e}nt te werk kan worden gegaan.

\section{Centrale onderzoeksvraag en deelvragen}

\section{Opdrachtgever}

\section{Werkomgeving en taken}


\chapter{Methode}

\section{Onderzoeksmethode}

\section{Informatie vergaren}

\section{Valideren van bevindingen}

\section{Geldigheid en betrouwbaarheid van bronnen}

\section{Projectmethode}

\section{Risicoanalyse}

\section{Kwaliteitsverwachtingen}


\chapter{Resultaten}

\section{Beoogd resultaat van de opdracht}


\chapter{Literatuur}


\bibliographystyle{plain}
\bibliography{References}


\chapter{Betrokkenen}

\section*{Afstudeerder}
\begin{table}[h]
\begin{tabular}{ll}
Naam & \name \\
Studentnummer & \studentnumber \\
E-mailadres & \email \\
Mobiel telefoonnummer & \mobilephone
\end{tabular}
\end{table}

\section*{Bedrijfsbegeleider}
\begin{table}[h]
\begin{tabular}{ll}
Naam bedrijf/organisatie &  \\
Studentnummer & \studentnumber \\
E-mailadres & \email \\
Mobiel telefoonnummer & \mobilephone
\end{tabular}
\end{table}

\section*{Opdrachtgever}

\end{document}