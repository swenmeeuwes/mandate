\documentclass{report}

% Packages
\usepackage[dutch]{babel}
\usepackage{datetime}
\usepackage{hyperref}
\usepackage{graphicx}

% Literals
\newcommand{\versionnumber}{0.1}

\newcommand{\name}{Swen Meeuwes}
\newcommand{\studentnumber}{088127}
\newcommand{\email}{0887127@hr.nl}
\newcommand{\mobilephone}{06 10 466 433}

\newcommand{\institution}{Rotterdam University of Applied Sciences}
\newcommand{\organisation}{\&ranj }

\renewcommand{\title}{Narratieve omgevingsbewerker}
\newcommand{\subtitle}{Sub title}

\begin{document}
\begin{titlepage}
        \centering
        \includegraphics[width=2cm]{Images/University}\par
        \vspace{4\baselineskip}
        {\Huge\title\par}
        %{\Large\subtitle\par}
        \par
        \includegraphics[width=3cm]{Images/Organisation}
        \vspace{4\baselineskip}
        \par
        {\Large\name\par}
        {Studentnummer: \studentnumber\par}
        \vfill
        {\hfill Versie: \versionnumber\par}
        {\hfill \today}
\end{titlepage}

\chapter*{Samenvatting}

\newpage

\tableofcontents

\newpage

\chapter{Inleiding}
\section{Werktitel}
Het verbeteren van tooling in narratieve spellen.

\section{Aanleiding}
%Moet naar over bedrijf
%Al sinds 1999 specialiseert \organisation zich in \emph{serious games}. Hoewel er meerdere definities van \emph{serious games} zijn beschrijven ze allemaal een spelomgeving waarin de focus niet uitsluitend op entertainment ligt \cite{Tarja2007}. Hiernaast komen bij het begrip \emph{serious games} woorden zoals 'training', 'simulatie', 'reclame' en 'educatie' naar boven \cite{Tarja2007}.
%\organisation past \emph{serious games} vooral toe in educatie, trainingen en simulaties door middel van een narratief \cite{websiteranj}.
%Om workflow te bevorderen wordt er gebruik gemaakt van een dialoog bewerker \cite{interviewivo}. Echter sluit deze tool niet meer aan bij de werkwijze van \organisation \cite{interviewivo} wat zorgt voor een obstakel in het ontwikkelproces.

Rond 2011 begon \organisation \emph{narrative games} te verwerken in hun \emph{serious games} om in een verhalende wijze gedragsverandering toe te passen. Om game designers deze narratieven te laten defini{\"e}ren zijn er 2 bewerkers gebouwd; een dialoog bewerker en een verhaal bewerker. De huidige versies van deze bewerkers zijn gemaakt met de \href{https://en.wikipedia.org/wiki/Apache_Flex}{Apache Flex SDK} in \href{http://www.adobe.com/devnet/actionscript/articles/actionscript3_overview.html}{ActionScript3}.
Over de jaren heen zijn de verwachtingen van de bewerkers veranderd, maar ze zijn niet tot weinig uitgebreid omdat de achterliggende softwarearchitectuur niet schaalbaar en houdbaar is. Tenslotte hebben steeds minder programmeurs binnen het bedrijf kennis van de code base achter de bewerkers.

\section{Belang}
De conclusie en de aanbevelingen die voort komen uit het afstudeerverslag zijn input voor de beslissingen van \organisation op het gebied van narratieve omgevingsbewerkers. Vanuit deze input zal het bedrijf een nieuwe omgeving opzetten waarin game designers hun narratieven kunnen defini{\"e}ren. Het is daarom ook belangrijk om de game designers bij dit onderzoek te betrekken. Verder zullen toekomstige games de output van deze narratieve bewerker moeten verwerken. Dit betekend dat de nieuwe omgeving ge{\"i}ntegreerd zal moeten worden met het game framework van \organisation en dat dus de game developers inspraak moeten over technische keuzes binnen het project. Tenslotte halen de klanten van \organisation ook profijt uit deze nieuwe omgeving. De nieuwe omgeving maakt de games stabieler en goedkoper om te produceren.

\section{Doelstelling}
Het bedrijf hoopt na 6 maanden te beginnen met het ontwikkelen van een nieuwe narratieve omgevings bewerker, zodat ze effici{\"e}nter en voor lagere kosten producten kunnen opleveren aan de klant. Hiervoor is het belangrijk om binnen de 6 maanden zoveel mogelijk kennis en ervaring te verzamelen. Verder kan er nagedacht worden over mogelijke oplossingen op problemen die voort komen uit het onderzoek zodat deze het ontwikkelproces later niet zullen hinderen.

\section{Probleemstelling}
Voor het defini{\"e}ren van dialogen in \emph{narrative games} gebruikt \organisation verouderde bewerkers die gemaakt zijn in \href{http://www.adobe.com/devnet/actionscript/articles/actionscript3_overview.html}{ActionScript3} met \href{https://en.wikipedia.org/wiki/Adobe_Flash_Builder}{Adobe Flash Builder} als \emph{integrated development environment}. Weinig programmeurs bij \organisation hebben kennis van \href{http://www.adobe.com/devnet/actionscript/articles/actionscript3_overview.html}{ActionScript3} en daarom wordt het steeds lastiger om op deze bewerkers voort te bouwen te onderhouden. Verder wekt de architectuur en schaalbaarheid van de bewerkers frustratie op bij de game developers en game designers omdat projecten verschillen in content en de bewerkers dit niet of met een moeizame work-around toelaten. Dit alles zorgt voor een daling in effici{\"e}ntie en bekommering op het gebied van innovatie.
De gewenste situatie is om te beschikken over een overzichtelijke narratieve omgevings bewerker met een schaalbare en houdbare architectuur. In deze narratieve omgevings bewerker kan er makkelijk nieuwe content worden toegevoegd door game designers. Vervolgens kunnen game developers dit integreren in de game.

\section{Centrale onderzoeksvraag en deelvragen}
\subsection{Centrale onderzoeksvraag}
Hoe kan er een schaalbare narratieve omgevings bewerker worden opgezet die inzetbaar is voor verschillende projecten en te hanteren is door game designers?

\subsection{Deelvragen}
% Keywords:
% Visual scripting
% Content construct
% Integratie game framework - Data format export
% Personalisatie per project (modulair)
% Schaalbaarheid en houdbaarheid
\begin{itemize}
\item Wie gaan de narratieve omgevings bewerker gebruiken en wat zijn hierbij hun eisen en wensen?
\item Hoe kan de bewerker inzichtelijk worden gemaakt voor de gebruikers, zodat zij deze kunnen hanteren?
\item Wat zijn de mogelijkheden om de data achter een narratief moduleren?
\item Welke mogelijke dataformaten zijn er om data van de narratieve omgevings bewerker op te slaan en te exporteren?
\end{itemize}

\section{Opdrachtgever}
De desbetreffende afstudeeropdracht wordt uitgevoerd bij \organisation gevestigd te Rotterdam. Het bedrijf houdt zich bezig met gedragsverandering door middel van \emph{gamification} en \emph{serious games}. En maakt deel uit van de creatieve sector.
Het bedrijf zelf bestaat uit ongeveer 40 medewerkers en maakt deel uit van een grotere firma; \&samhoud. Enkele producten van \organisation zijn: \href{https://ranj.com/products#knowledge-knock-out}{Knowledge Knock-out}, \href{https://ranj.com/projects/corporate/development#mission-zhobia}{Mission Zhobia} (voor Peace Nexus), \href{https://ranj.com/projects/corporate/development#appie-aandeel}{Appie aandeel} (voor Albert Heijn), \href{https://ranj.com/projects/education#pinpin}{PinPin} (voor Rabobank).
De visie van \organisation luidt: ``Together we build a brighter future''.
Hiernaast heeft \organisation 4 core values:
\begin{description}
\item[Playfulness] plezier en een goed humeur hebben. Het leven als een spel zien.
\item[Intensity] passie om uit te blinken.
\item[Authenticity] durf jezelf te zijn, durf anders dan andere te zijn.
\item[Friendship] je kunt op elkaar rekenen.
\end{description}


\section{Werkomgeving en taken}


\chapter{Methode}

\section{Onderzoeksmethode}

\section{Informatie vergaren}

\section{Valideren van bevindingen}

\section{Geldigheid en betrouwbaarheid van bronnen}

\section{Projectmethode}

\section{Risicoanalyse}

\section{Kwaliteitsverwachtingen}


\chapter{Resultaten}

\section{Beoogd resultaat van de opdracht}


\chapter{Literatuur}


\bibliographystyle{plain}
\bibliography{References}


\chapter{Betrokkenen}

\section*{Afstudeerder}
\begin{table}[h]
\begin{tabular}{ll}
Naam & \name \\
Studentnummer & \studentnumber \\
E-mailadres & \email \\
Mobiel telefoonnummer & \mobilephone
\end{tabular}
\end{table}

\section*{Bedrijfsbegeleider}
\begin{table}[h]
\begin{tabular}{ll}
Naam bedrijf/organisatie & \organisation \\
Naam bedrijfsbegeleider & Ivo Swartjes \\
E-mailadres & ivo@ranj.nl \\
Telefoonnummer & +31 (0) 10 21 23 101 \\
Functie/ rol & Technical Team Lead \\
Bezoekadres locatie organisate & Lloydstraat 21m \\ 
 & 3024 EA Rotterdam \\
 & The Netherlands \\
Website organisatie & \url{https://ranj.nl/}
\end{tabular}
\end{table}

\section*{Opdrachtgever}
\begin{table}[h]
\begin{tabular}{ll}
Naam bedrijf/organisatie & \organisation \\
Naam bedrijfsbegeleider &  \\
E-mailadres &  \\
Telefoonnummer & +31 (0) 10 21 23 101 \\
Functie/ rol &  \\
Bezoekadres locatie organisate & Lloydstraat 21m \\ 
 & 3024 EA Rotterdam \\
 & The Netherlands \\
Website organisatie & \url{https://ranj.nl/}
\end{tabular}
\end{table}

\end{document}